ich habe ihren Nachruf auf den Arzt Dr.~Fritz Hochstetter mit Interesse und Anspannung gelesen. In
dem Artikel beleuchten Sie sehr ausführlich die langjährigen Verdienste von Hochstetter als
ehemaligen Leiter des Psychiatrischen Landeskrankenhauses in Zwiefalten. Ich habe mich tierisch über
ihren sehr positiv gehaltenen Artikel aufgeregt, denn er hatte nicht nur gute Seiten~…

Mein Name ist Hermann Mauser. Ich bin Grundschullehrer für Deutsch, Rechnen,
Heimatkunde und Sachunterricht und erforsche gerne in der Umgebung.
Ich kenn’ die Gegend wie kein zweiter. Mein Wohnort ist Buttenhausen, schon immer gewesen.
Infolge eines Leichenfundes in der Lehmkammerhöhle
und den Ermittlungen zu dem Fall habe ich mich in den letzten Wochen intensiv mit der Vergangenheit
beschäftigt. Bei der Leich’ handelt es sich, wie wir erst vor Kurzem durch eine Rekonstruktion des
Gesichtes wissen, um Dr. Jürgen Schuhmacher, den Leiter von Grafeneck.

Durch meine Ermittlungen bin ich auf Dr. Fritz Hochstetter aufmerksam geworden. Ich habe dann kurz
vor Hochstetters Tod noch mit ihm gesprochen. Nach meinen Informationen hatte er mit Schuhmacher
gemeinsame Sache gemacht. Hochstetter hat die Leute aus der Umgebung und seiner Anstalt an
Schuhmacher
ausgeliefert. Und Schuhmacher hat sie dann in dem alten Schuppen unweit vom Schloss Grafeneck
vergast.

Hochstetter war, wie sie schon in ihrem Artikel sagten, der ehemalige Leiter des Psychiatrischen
Landeskrankenhauses in Zwiefalten zu Zeiten des Nationalsozialismus. Allerdings erwähnen Sie mit
keinem Wort, dass er dort die Atteste für die Behinderten ausgestellt hat und dadurch, im Rahmen der
Aktion T4, für den Tod von Hunderten Menschen verantwortlich war. Darunter auch meine ältere
Schwester. Genau 20 Jahre alt ist sie geworden. Dann wurde sie abgeholt. Mein Vater, der damals
Polizist war, konnte es nicht verhindern. Dieser Vorfall hat alles verändert. Meine Mutter hat sich
zwei Tage später umgebracht! Ich war erst acht, als es passierte. Ich hab sie gefunden. Blaues,
verzerrtes Gesicht hatte sie. Ich hab sie gefunden mit dem Strick um den Hals. Und das alles wegen
dem „guten“ Fritz Hochstetter~…

Am Ende des Zweiten Weltkriegs, als langsam bekannt wurde, was da in Grafeneck vor sich ging und
wofür die grauen Busse mit den weißbemalten %% Seite 151
Fensterscheiben genutzt wurden,
versuchte Hochstetter zusammen mit Dr. Jürgen Schuhmacher zu fliehen, um einer Strafe zu entgehen!
Glücklicherweise hatten sie eine Reifenpanne – die Gerechtigkeit Gottes. Die beiden gingen in den
„Pflug“ und wurden dort von meinem Vater erkannt. Vater, Gottfried Staudinger und ein Dritter
konnten die Flucht dieser Verbrecher nicht geschehen lassen. Mein Vater hat sich immer für die
Gerechtigkeit eingesetzt, müssen Sie wissen. Also haben sie sich zusammen geschlossen und
Hochstetter und Schuhmacher in den Wald unweit von Hundersingen, oberhalb der Lehmkammerhöhle
geführt. Nach meinen Information kam es im Wald zu einer tödlichen Inszenierung, bei der Schuhmacher
erschossen wurde. Ich weiß nicht, wer geschossen hat und Hochstetter wollte es mir bei meinen zwei
Besuchen auch nicht sagen. Aber Vater sagte mal so etwas wie „ein Lump soll den anderen richten“,
von dem her glaube ich, dass sie Hochstetter gezwungen haben Schuhmacher zu erschießen. Mein Vater
hat dem Schuhmacher damals noch einen Anzug und eine Krawatte angezogen und ein Kreidekreuz auf den
Rücken gezeichnet. Auf dass er so sterben möge wie seine zahlreichen Opfer. So wie es aussah, haben
sie den Hochstetter dann weggeschickt. Und nun habe ich nach 50 Jahren die Leich’ von Schuhmacher
gefunden.

Obwohl diese Selbstjustiz mit dem Mord eines Menschen sehr weit geht, muss man aber doch versuchen,
meinen Vater zu verstehen. Es waren beides Nazis, die Hunderte auf dem Gewissen hatten. Zudem wollte
sie einfach so abhauen, als wäre nichts gewesen~…

Das hat Hochstetter dann auch sein restliches Leben gemacht. Man konnte ihm ja bei den
Ärzteprozessen nach 1945 nichts nachweisen. Er stand vor Gericht und hat alles geleugnet! Hat
einfach so weiter gelebt und sogar seine Stelle als Leiter in Zwiefalten behalten, obwohl er
wiederholt gegen die Grundsätze eines Arztes verstoßen hat, indem er die Atteste auf unwertes Leben
ausgestellt hat. Nach seiner Pensionierung lebte er dann einfach so weiter. Er hat sich dann unweit
von Grafeneck in Hundersingen in einem Neubau niedergelassen. Er ist dann reich und einsam an einem
Herzanfall gestorben.

Als ich ihn auf seine dunkle Vergangenheit angesprochen habe, leugnete er alles. Auch das er meine
Schwester auf dem Gewissen hat, brachte ihn nicht zum Reden. Wollt’ seine Schuld wohl mit ins Grab
nehmen. Das ist dem weißhaarigen gebrechlichen Mann dann mit 89 Jahren, %% Seite 149 unten
wie Sie schreiben, auch gelungen.

Durch die Leich’ von Schuhmacher kam das alles erst ins Rollen. Durch ihn bin ich auf den Fall
aufmerksam geworden. Dadurch hat sich die Wahrheit bezüglich Schuhmacher und Hochstetter gezeigt.
Dies müssen Sie unbedingt richtigstellen, das ist mir sehr sehr wichtig. Da sonst Dr. Fritz
Hochstetter in einem Bild erscheint, dem er nicht entspricht!
